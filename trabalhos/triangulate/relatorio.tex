\documentclass{article}
\usepackage[utf8]{inputenc}

\title{Trabalho 1 - Triangulação de polígono}
\author{Gustavo Higuchi}
\date{\today}

\usepackage{natbib}
\usepackage{graphicx}
\usepackage{amssymb}
\usepackage{amsthm}
\usepackage{amsmath}
\usepackage{color}   %May be necessary if you want to color links
\usepackage[portuguese, ruled, linesnumbered]{algorithm2e}
\usepackage{inconsolata}

\usepackage{mathtools}
\DeclarePairedDelimiter\ceil{\lceil}{\rceil}
\DeclarePairedDelimiter\floor{\lfloor}{\rfloor}

% usado para linkar cada section na tabela de conteúdo com a respectiva
% página no documento
\usepackage{hyperref}
\hypersetup{
    colorlinks,
    citecolor=black,
    filecolor=black,
    linkcolor=black,
    urlcolor=black,
    linktoc=all
}

%o começo do documento
\begin{document}

% compila o título
\maketitle
\newpage
% compila a tabela de conteúdos
\tableofcontents
\newpage
\section{Resumo}
\hspace*{15pt}O problema escolhido para resolver foi a \textit{triangulação de polígonos} e a
solução implementada foi dividir o polígono em sub-polígonos monotônicos e,
em seguida, triangular esses subpolígonos, resultando numa triangulação do polígono
original. O algoritmo foi implementado em \textit{Python}, utilizando uma abordagem
que primeiro divide o polígono em polígonos \textit{y-monotone} para então
triangular cada sub-polígono em tempo linear, o algoritmo roda em tempo $O(n\log n)$.

\section{Introdução}
\hspace*{15pt}Em geometria, uma triangulação é uma subdivisão de um objeto geométrico em
\textit{simplexos} \cite{wikiTriang}. Neste trabalho é feito uma triangulação
de polígonos simples de duas dimensões. Para resolver o problema foi utilizada
a solução apresentada no livro \textit{Computational Geometry Algorithms and Applications}
\cite{bergBook}. O algoritmo apresentado neste trabalho utiliza uma ordenação e
por isso é limitado inferiormente a rodar em tempo $O(n \log n)$ e é descrito
na secção 4.

\section{Descrição do problema}
\hspace*{15pt}O problema da triangulação de polígonos consiste em dividir um polígono com diagonais
que formem um triangulo, sem que seja adicionado um novo vértice.
Para um polígono de $n$ vértices, uma triangulação desse polígono teria exatamente
$n-2$ triangulos.\\
\\
\hspace*{15pt}Para um polígono simples e não-convexo, basta ligar um vértices com todos os
outros vértices exceto seus vizinhos, então teríamos uma triangulação em tempo
linear com relação à entrada. Porém, em casos que o polígono é convexo, é preciso
verificar se a diagonal inserida não está fora do polígono.\\
\\

\section{O algoritmo}
\hspace*{15pt}O algoritmo neste trabalho utiliza uma abordagem para dividir o polígono simples
em sub-polígonos \textit{y-monotônicos}, i.e. uma linha horizontal intersecta a
fronteira da esquerda e a fronteira da direita, ou não intersecta nenhuma aresta
do polígono. Isto é feito utilizando uma \textit{sweep line} que passa do vértice
mais acima até o vértice mais abaixo, portanto é necessário organizar os vértices
em uma fila de prioridades pela coordenada $y$, no caso de mesma coordenada $y$,
o vértice mais a direita é usado como critério de prioridade. \\
\hspace*{15pt}A estrutura de dados utilizada é a mesma que do livro da disciplina,
\textit{Doubly Connneted Edge List} ou \textit{DCEL} e é descrita no livro de
Berg et al. \cite{bergBook}, com algumas modificações específicas da implementação
que fiz e é basicamente assim:
\begin{enumerate}
    \item Cada vértice possui apontadores para o próximo vértice e o anterior,
    além de suas coordenadas e angulo inicial.
    \item Cada face possui um apontador para uma aresta interna
    \item Cada aresta possui um apontador para o vértice de origem, o vértice de
          destino, o vértice helper (utilizado no algoritmo de divisão em sub-polígonos
          y-monotônicos), a próxima aresta e a anterior, assim como um apontador para a
          face que ela está e sua aresta \textit{gêmea} que é denominada de \textit{twin}.
    \item O polígono possui uma lista de vértices, uma lista de arestas e uma lista de
          faces.
\end{enumerate}
\hspace*{15pt}O algoritmo implementado é tal como aquele apresentado
por Berg et al \cite{bergBook}, com a exceção de que a triangulação de polígonos
y-monotônicos é feita diferente e será descrito em seguida.\\
\hspace*{15pt}Depois de dividido, a triangulação dos sub-polígonos
\textit{y-monotônico} é trivial e foi utilizado o algoritmo \textit{ear clipping},
que insere uma diagnoal entre o início de uma aresta para o destino de outra aresta,
caso o angulo formado pelas duas arestas for menor que $\pi$. O algoritmo
\textit{triangulate} descrito abaixo descreve a implementação feita. \\
\begin{algorithm}[H]
  \SetAlgoLined
  \Entrada{Um polígono $P$ subdividido em polígonos monotônicos}
  \Saida{O polígono $P$ subdividido em triângulos}
  \Inicio{
  $Q\gets \text{lista de faces de P}$\\
  \Para{cada elemento de Q}{
    $f \gets \text{primeiro elemento de Q}$\\
    $\text{remove o primeiro elemento de Q}$\\
    \Se{\text{f não for um triangulo}   }{
        $v_i \gets \text{destino da aresta inicial de f}$\\
        $v_j \gets \text{origem da aresta anterior de $v_i$}$\\
        $\text{$ear_clipping(P, v_i$, $v_j)$}$\\
        $\text{insere em Q a face de }$\\
    }
    }
  }
  \label{alg1}
  \caption{triangulate}
\end{algorithm}
\hspace*{15pt}A função \texttt{ear\_clipping()} insere uma diagonal entre dois
vértices $v_i$ e $v_j$, apenas uma manipulação de ponteiros da aresta anterior
e posterior dos dois vértices, executando em tempo constante. A inserção dessa
diagonal divide o polígono em dois sub-polígonos, criando uma face nova. Essa face
aponta para a aresta \textit{gêmea} como sendo sua aresta incial.
\\
\subsection{Complexidade e corretude}
\hspace*{15pt}A construção da fila de prioridades é feita com uma ordenação
baseada em quick sort, então executa em tempo $O(n \log n)$, e a triangulação
do poligono é feita em tempo linear, levando em conta que o laço da linha 3
executa no máximo o número de triangulos que o polígono pode possuir, ou seja
$n-2$ vezes. Assim o algoritmo inteiro ficaria limitado em tempo $O(n \log n)$.
\\
\\
\hspace*{15pt}A cada iteração do laço, uma face de $Q$ é removida, e
caso a face não seja um triangulo, transforma a face em um triangulo e insere uma
nova face em $Q$, então o algoritmo para quando houver apenas triangulos em $Q$.
\newpage
\bibliography{geometria}
\bibliographystyle{ieeetr}

\end{document}
