\documentclass{article}
\usepackage[utf8]{inputenc}

\title{Trabalho 3 - Consultas Retangulares}
\author{Gustavo Higuchi}
\date{\today}

\usepackage{natbib}
\usepackage{graphicx}
\usepackage{amssymb}
\usepackage{amsthm}
\usepackage{amsmath}
\usepackage{color}   %May be necessary if you want to color links
\usepackage[portuguese, ruled, linesnumbered]{algorithm2e}
\usepackage{inconsolata}
\usepackage{listings}
\usepackage{color}
 
\definecolor{codegreen}{rgb}{0,0.6,0}
\definecolor{codegray}{rgb}{0.5,0.5,0.5}
\definecolor{codepurple}{rgb}{0.58,0,0.82}
\definecolor{backcolour}{rgb}{0.96,0.96,0.96}
 
\lstdefinestyle{mystyle}{
    backgroundcolor=\color{backcolour},   
    commentstyle=\color{codegray},
    keywordstyle=\color{blue},
    numberstyle=\tiny\color{codegray},
    stringstyle=\color{codepurple},
    basicstyle=\footnotesize\ttfamily,
    breakatwhitespace=false,         
    breaklines=true,                 
    captionpos=b,                    
    keepspaces=true,                 
    numbers=left,                    
    numbersep=5pt,                  
    showspaces=false,                
    showstringspaces=false,
    showtabs=false,                  
    tabsize=2
} 
\lstset{style=mystyle}

\newtheorem{definicao}{Definição}
\theoremstyle{definition} 


\usepackage{mathtools}
\DeclarePairedDelimiter\ceil{\lceil}{\rceil}
\DeclarePairedDelimiter\floor{\lfloor}{\rfloor}

% usado para linkar cada section na tabela de conteúdo com a respectiva
% página no documento
\usepackage{hyperref}
\hypersetup{
    colorlinks,
    citecolor=black,
    filecolor=black,
    linkcolor=black,
    urlcolor=black,
    linktoc=all
}

%o começo do documento
\begin{document}

% compila o título
\maketitle
\newpage
% compila a tabela de conteúdos
\tableofcontents
\newpage
\section{Resumo}
\hspace*{15pt} O problema de consultas retangulares consiste em realizar
uma busca por um subconjunto de segmentos no plano de forma eficiente. 
Este documento descreve uma solução para este problema realizando um 
pré-processamento de tempo $O(n\log n)$ e utilizando uma estrutura de
dados que ocupa $O(n\log n)$, para então realizar cada consulta com um 
custo computacional de $O(\log^2 n + k)$, onde $n$ é o número de segmentos
no plano e $k$ é o número de segmentos dentro da região retangular que 
se deseja saber os segmentos contidos nele. A solução do problema apresentada
neste documento supõe que os segmentos estejam disjuntos, ou seja, não
se intersectam.

\section{Introdução}
\hspace*{15pt} O problema das consultas retangulares geralmente é utilizado
por aplicações que fazem uso de localização dentro de um mapa, como
um serviço de GPS, que exibe apenas uma pequena porção de um mapa em uma
tela retangular. Portanto, deseja-se apenas uma pequena porção de um mapa
para exibição. O problema pode ser dado da seguinte forma:\\
\begin{definicao}
    Dado um conjunto de segmentos $S$ no plano e um sub-espaço retangular 
    do plano dado por $[x:x']x[y:y']$, encontrar o subconjunto de segmentos
    dentro deste sub-espaço retangular de forma eficiente.
\end{definicao}
\hspace*{15pt}Realizar uma verificação para cada segmento se está ou não 
dentro de um sub-espaço, que chamaremos daqui em diante de \textit{janela}, 
seria impraticável. A solução que será discutida neste documento será para 
um conjunto de segmentos \textbf{disjuntos}, ou seja, que não se intersectam. 


\section{Descrição da solução}
\hspace*{15pt} O problema das consultas retangulares pode ser dividido em duas
partes, os segmentos que possuem algum ponto \textbf{dentro} da janela, e os
segmentos que intersectam duas fronteiras da janela.

\subsection{2D Range Tree}
\hspace*{15pt} Para a primeira parte, é usado uma estrutura de dados chamada
\textit{2D Range Tree} que essecialmente realiza uma consulta por um intervalo 
retangular composta de duas outras buscas por intervalos de uma dimensão. A 
estrutura de dados é composta de pontos no plano e pode ser descrita como uma 
árvore onde cada nó $N$ possui sua chave baseada na coordenadas x de cada ponto, e com
um ponteiro para uma estrutura de dados associadas. Essa estrutura associada é 
uma árvore binária balanceada que representa a união das sub-árvores esquerda e 
direita de $N$, porém com sua chave baseada na coordenada y dos pontos. A busca 
nessa estrutura é feita a partir de um intervalo $[x:x']$ e em seguida, na estrura
associada buscando um intervalo $[y:y']$, os nós que pertencem à estrutura associada
são chamados de \textit{subconjunto canônico do nó}. Basicamente, funciona da 
seguinte forma:
 \begin{itemize}
    \item Primeiro, busca pelo nó divisor, i.e. nó cuja chave $k \in [x:x']$ 
    \item A partir do nó divisor, para todo nó na sub-árvore esquerda, reportar
        todos os nós à direita.
    \item De forma semelhante, fazer para os nó na sub-árvore direita.
    \item Fazer até as folhas que representam os extremos do intervalo,
         $x$ e $x'$, respectivamente
 \end{itemize}

\hspace*{15pt} Então, com essa estrutura, podemos reportar os segmentos que possuem
ao menos um endpoint dentro da janela. Realizando o cuidado de não reportar mais 
de uma vez.

\subsection{Segment Tree}
\hspace*{15pt} Com os segmentos que possuem algum endpoint dentro da janela reportados,
falta apenas os segmentos que intersectam duas fronteiras da janela. Para esse tipo
de busca, será necessário uma estrutura de dados chamada \textit{segment tree}, ou
\textit{árvore de segmentos}. Esta estrutura é baseada em \textit{intervalos elementares}.
O intervalos elementares podem ser descritos da seguinte forma:
\begin{definicao}
    Seja $p_1, p_2, p_3, ..., p_n$ uma lista de endpoints distintos ordenados dao menor
    para o maior, os intervalos elementares são definidos pelo conjunto de intervalos
    fechados $[p_i:p_i]$ e intervalos abertos $(p_i:p{i+1})$, cujas pontas representam
    o infinito, ou seja, os intervalos elementares destes pontos são:
    $(-\infty: p_1), [p_1:p_1], (p_1:p_2), [p_2:p_2], ..., (p_{n-1}:p_n), [p_n:p_n], (p_n:+\infty)$
\end{definicao}
\hspace*{15pt} Assim, constrói-se uma árvore binária balanceada cujas folhas representam
os intervalos elementares, denotados de $Int(v)$. 

\hspace*{15pt} Com a árvore de intervalos construida, insere os segmentos tal que\\
\mbox{$Int(v) \subseteq [x:x'] \textbf{ e } Int(Pai(v)) \not\subseteq [x:x']$}.

\hspace*{15pt} Sendo assim, a estrutura fica assim:
\begin{itemize}
    \item O esqueleto da árvore de segmentos é uma árvore binária balanceada $\tau$,
        As folhas de $\tau$ correspondem a um intervalo elementar induzido pelos
        endpoints ordenados.
    \item Os nós internos de $\tau$ correspondem aos intervalos que são a união dos
        intervalos elementares do subconjunto canônico.
    \item Cada nó ou folha $v$ em $\tau$ guarda o segmento $s$ tal que \\
        \mbox{$Int(v) \subseteq [s_x:s_x'] \textbf{ e } Int(Pai(v)) \not\subseteq [s_x:s_x']$},
        onde $s_x$ e $s_x'$ correspondem às coordenadas x do segmento $s$
\end{itemize}

Os segmentos são armazenados em ordem em uma árvore binária balanceada 
com base na ordem vertical deles.\\
De forma semelhante, é feito uma segunda estrutura do mesmo tipo, porém 
trocando as coordenadas.

\hspace*{15pt} Esta abordagem é suficiente para se obter consultas retangulares eficientes.

\section{O algoritmo}
\subsection{Estruturas auxiliares}
\hspace*{15pt} O algoritmo foi desenvolvido em Python, utilizei classes 
para representar as estruturas de dados, em ordem, temos:

\hspace*{15pt}O Ponto possui um ponteiro para o segmento para o qual 
pertence, além de suas coordenadas. 
\lstinputlisting[language=Python]{classPoint.py}

\hspace*{15pt}O segmento que instancia um objeto que possui um inteiro
que representa o índice $[1:n]$, um ponteiro \texttt{upper}
para o endpoint com a coordenada $y$ maior e o \texttt{lower} para
o endpoint com a coordenada $y$ menor, além de uma referência para
qual endpoint está à esquerda e qual está à direita e um booleano
que indica se o segmento já foi reportado.
\lstinputlisting[language=Python]{classSegment.py}

\hspace*{15pt} O intervalo possui a inteiro \texttt{left} que guarda
a menor coordenada e um inteiro \texttt{right} que guarda a maior
coordenada do intervalo.
\lstinputlisting[language=Python]{classInterval.py}


\subsection{As funções}
A função principal do código ficou assim:
\lstinputlisting[language=Python]{windowQuery.py}
\hspace*{15pt} Por questões de simplicidade de código, optei utilizar uma 
função interna do Python que ordena uma array, para ordenar os endpoints 
dos segmentos, e estou assumindo que executa em tempo $O(n\log n)$.

\hspace*{15pt} Para construir a Range Tree de duas dimensões, divido
o conjunto de Pontos $P$ baseado na mediana em tempo linear $O(n)$ e 
executo essa operação $O(\log n)$ vezes, totalizando $O(n\log n)$ 
operações para construir a Range Tree.
\lstinputlisting[language=Python]{rangeTree.py}

\hspace*{15pt} Como cada ponto guarda um ponteiro para o segmento que ele
pertence, a função \texttt{query2DRangeTree()} funciona como explicado 
anteriormente, fazendo uma busca por endpoints que estão dentro da janela
e reportando os segmentos dos pontos, executando um total de $O(\log^2n + k)$
operações\cite{bergBook}. 
\lstinputlisting[language=Python]{query2DRangeTree.py}

\hspace*{15pt} Para construir a árvore com os intervalos elementares, verifica
se o intervalo do nó $Int(v)$ é fechado, semi-fechado à esquerda, semi-fechado à direita
ou aberto, para então instanciar o intervalo, que leva os argumentos 
\texttt{Interval(esquerda, direita, fechado, lado\_aberto)}, onde \texttt{esquerda}
é o ponto à esquerda do interval, \texttt{direita} é o ponto à direita do intervalo,
\texttt{fechado} é um booleano e \texttt{lado\_aberto} é um inteiro que 
define o lado que está aberto no intervalo semi-fechado. Então, levando em conta
que são $2n + 1$ folhas e $O(\log n)$ nós internos, temos um total de 
$O(\log n)$ operações.
\lstinputlisting[language=Python]{intervalTree.py}

E para inserir cada segmento na árvore, percorre a árvore de intervalos
criada e insere o segmento nó mais alto possível, executando $O(\log n)$ operações 
para cada segmento, totalizando $O(n\log n)$ operações\cite{bergBook}.
\lstinputlisting[language=Python]{segmentTree.py}

A função \texttt{querySegmentTreeVertical()}
realiza as buscas por segmentos que intersectam fronteiras verticais
da janela, i.e. baseado no eixo x. Dado um segmento de busca $q := q_x \texttt{x} [q_y:q_y']$, 
percorre a árvore até sua folha, para cada nó no caminho, reporta todos
os segmentos possuem intersecção com $q$, executando $O(\log n)$ operações
para caminhar na árvore, mais $O(\log n)$ para caminhar na estrutura associada
e $O(1 + k)$ para reportar os segmentos, onde $k$ é o número de segmentos 
intersectado por $q$, totalizando um tempo de $O(\log^2n + k)$ \cite{bergBook}. 
Trocando as coordenadas, a função \texttt{querySegmentTreeHorizontal()}
funciona de forma semelhante.
\lstinputlisting[language=Python]{querySegmentTreeVertical.py}

\newpage
\bibliography{relatorio}
\bibliographystyle{ieeetr}

\end{document}
