

\section{Algoritmo Incremental baseado em flips}

O algoritmo que será apresentado é baseado na inserção dos pontos de $S$ um de cada vez.
Para cada novo ponto $p$ inserido, é modificado a configuração dos tetraedros presentes nas vizinhanças 
de $p$ através de $flips$ que foram descritos acima. 

O algoritmo baseado em flips é relativamente mais demorado que o algoritmo que 
gera um buraco, apresentado por \cite{Engwirda2015}, porém é mais simples de implementar.

\subsection{Duas dimensões}

O primeiro algoritmo baseado em flips foi feito para construir a TD em duas dimensões.
Desenvolvido por Lawson \cite{Lawson1977} que provou que, começando com uma 
triangulação arbitrária de um conjunto $S$ de pontos no plano, ``flipping" 
as arestas pode transformar essa triangulação em qualquer outra triangulação de S,
incluindo uma triangulação de Delaunay. 

Basicamente, primeiro o algoritmo busca o triangulo $\tau$ que contém o ponto
$p$ recem inserido e então o divide em três novos triangulos que contém o 
ponto $p$. Em seguida, os novos triangulos gerados são testados com os triangulos
opostos em relação à $p$. Caso algum triangulo não seja um triangulo de Delaunay,
então a aresta compartilhada pelos triangulos é ``flipado" e os dois novos triangulos
gerados desta forma serão testados depois. 

\subsection{Três dimensões}

Apesar do conceito de ``flipping" generalizar para três dimensões, o algoritmo 
de Lawson \cite{Lawson1977} não generaliza, provado por Joe \cite{Joe1989}.
Mais tarde, porém, Joe \cite{Joe1991} prova que dado uma $TD(S)$ e um ponto 
$p$ em $\mathbb{R}^3$, sempre existe pelo menos uma sequencia de $flips$ para 
construir a $TD(S \cup \{p\})$. Neste caso, ainda haverá facetas que não são
Delaunay que não se pode aplicar um \textit{flips}, mas que sempre haverá algum
$flip$ possível de se executar tal que o algoritmo continue executando.

Então, com isso é possível adaptar o algoritmo de duas dimensões para que 
se gere o $TD(S)$ em três dimensões. Daí temos 
\begin{definicao}
    Seja $S$ um conjunto de pontos em $\mathbb{R}^3$, e $\mathcal{T}$ a $TD(S)$,
    o algoritmo \texttt{InsertOnePoint()} é necessário para manter $\mathcal{T}$
    dentro dos critérios de Delaunay quando um ponto $p$ é inserido.    
\end{definicao}

\begin{algorithm}
\DontPrintSemicolon % Some LaTeX compilers require you to use \dontprintsemicolon instead
\KwIn{Uma $TD(S) \mathcal{T}$ em $\mathbb{R}^3$, e um novo ponto $p$ para inserir}
\KwOut{$\mathcal{T}^p = \mathcal{T} \cup \{p\}$}
$\tau \gets Walk(p)$\\
$\text{insere p em }\tau\text{ com um flip14}$\\
$\text{empilha os 4 tetraedros que se formaram}$\\
\While{$\text{pilha não for vazia}$}{
  $\tau = \{p,a,b,c\} \gets \text{desempilha um tetraedro}$\\
  $\tau_a = \{a,b,c,d\} \gets \text{tetraedro adjacente de }\tau\text{ que contém abc como faceta}$\\
  \uIf{$\text{d está dentro da circunsfera de }\tau$}{
      $Flip(\tau, \tau_a)$\\
    }
  }
\caption{\texttt{InsertOnePoint($\mathcal{T}, p$)}}
\label{alg1}
\end{algorithm}

O algoritmo é uma adaptação de \cite{Joe1991}. Como em $\mathbb{R}^2$, o ponto $p$
é primeiro inserido em $\mathcal{T}$ com um $flip$ (neste caso, $flip14$), e os 
novos tetraedros são testados para garantir que ainda são Delaunay. A sequencia de 
$flips$ necessários é controlado pela pilha que contém todos os tetraedros não
testados. Toda vez que $flip$ é executado, o tetraedro gerado é empilhado. O algoritmo
para quando todos os tetraedros incidentes a $p$ são Delaunay, o que significa que 
a pilha está vazia.

O método \texttt{Flip()} necessita de cuidado especial, ao contrário do caso para duas
dimensões, existem diferentes tipos de $flips$ para diferentes situações. Basicamente,
testar para cada face que não possue o ponto inserido $p$, se são localmente 
Delaunay.  

Seja $\tau = \{p,a,b,c\}$ o próximo tetraedro a ser desempilhado. Para que 
$\tau$ seja Delaunay, é necessário apenas uma comparação com o tetraedro 
$\tau_a = \{a,b,c,d\}$ que não possue $p$ e é incidente à face $abc$. Se o
vértice $d$ estiver dentro da circunsfera de $\tau$. Caso esteja, realiza
uma operação diferente dependendo do tipo de geometria que de $\tau$ e 
$\tau_a$.

Observando do ponto de vista do ponto $p$, temos três possibilidades: vemos
uma face de $\tau_a$; vemos duas faces de $\tau_a$; ou vemos três faces de
$\tau_a$. Joe \cite{Joe1989} não só provou que o algoritmo incremental baseado
em $flip$ que funciona em 3 dimensões, mas também provou que funciona para
casos degenerados. Disso temos os seguintes casos que podem ocorrer:

\begin{Casos}
    \item apenas uma face de $\tau_a$ é visível, e portanto a união de 
    $\tau$ e $\tau_a$ é um poliedro convexo. Neste caso, executa um 
    $flip23$.
    \item duas faces de $\tau_a$ são visíveis, e portanto a união de 
    $\tau$ e $\tau_a$ é não-convexo. \textbf{Se} existir um tetraedro $\tau_b = abpd$
    em $\mathcal{T}^p$ tal que a aresta $ab$ é compartilhada por $\tau$,
    $tau_a$ e $\tau_b$, então um $flip32$ é executado. \textbf{Caso contrário},
    nenhum $flip$ é executado. A face $abc$ que não é Delaunay será corrigida
    por outro $flip$ executado em um tetraedro adjacente, como \cite{Joe1991}
    prova.
    \item o segmento de reta $pd$ intersecta diretamente uma aresta de $abc$ 
    (assuma que seja $ab$, por exemplo). Então, os vértices $p$, $d$, $a$ e 
    $b$ são coplanares. Um $flip23$ criaria um tetraedro ``achatado" $abdp$,
    é possível se, e somente se, $\tau$ e $\tau_a$ estão na $config44$ com dois
    outros tetraedros $\tau_b$ e $\tau_c$, tal que a aresta $ab$ é incidente a
    $\tau$, $\tau_a$, $\tau_b$ e $\tau_c$. Uma vez que os quatro tetraedros estão
    na $config44$, um $flip44$ pode ser executado, caso contrário, nenhum $flip$
    é executado.
    \item o vértice $p$ está diretamente em cima de uma aresta da face $abc$ (assuma 
    que seja a aresta $ab$). Portanto, $p$ está no mesmo plano que abc, mas também
    está no mesmo plano que $abd$. O tetraedro $abcp$ é obviamente achatado. A única 
    razão para tal caso é porque $p$ foi inserido diretamente em uma aresta de $\mathcal{T}$,
    e o $flip14$ que divide o tetraedro, criou dois tetraedros achatados que contém $p$.
    Neste caso, é suficiente executar um $flip23$ em $\tau$ e $\tau_a$. Um tetraedro 
    achatado ainda será criado, porém será deletado com um outro $flip$.
    \item existe um ponto exatamente no mesmo lugar que o ponto $p$ que está sendo inserido,
    este caso é trivial, no método \texttt{Walk()}, testar se o ponto $p$ não está
    dentro de uma tolerancia de distancia dos vértices, caso esteja, não insere $p$.
    \item o ponto $p$ cai exatamente na superfície de uma circunsfera de $\mathcal{T}$,
    este caso também é trivial e não necessita de nenhuma operação.
    \item o ponto $p$ cai em uma face de um tetraedro de $\mathcal{T}$. Neste caso, 
    não necessita de nenhuma operação, uma vez que um $flip14$ será executado e criará
    um tetraedro achatado que será deletado mais tarde. 
\end{Casos}

\begin{algorithm}[H]
\DontPrintSemicolon % Some LaTeX compilers require you to use \dontprintsemicolon instead
\KwIn{Dois tetraedros adjacentes $\tau$ e $\tau_a$}
\KwOut{um flip é executa, ou não}
\uIf{\text{caso 1}}{
    $flip23(\tau, \tau_a)$\\
    $\text{empilha tetraedra $pabd$, $pbcd$ e $pacd$}$
}
\Else{
    \uIf{\text{caso 2 }\textbf{and}\text{ $\mathcal{T}^p$ tem tetraedro $pdab$} }{
        $flip32(\tau, \tau_a, pdab)$\\
        \text{empilha $pacd$ e $pbcd$}
    }
    \Else{
        \uIf{\text{caso 3 }\textbf{and}\text{ $\tau$ e $\tau_a$ estão em $config44$ com $\tau_b$ e $\tau_c$}}{
            $flip44(\tau, \tau_a, \tau_b, \tau_c)$\\
            \text{empilha os quatro tetraedros criados}
        }
        \Else{
            \uIf{\text{caso 4}}{
                $flip23(\tau, \tau_a)$\\
                \text{empilha o tetraedro $pabd$, $pbcd$ e $pacd$}
            }
        }
    }
}
\caption{\texttt{Flip($\mathcal{T}, p$)}}
\label{alg1}
\end{algorithm}

O algoritmo \texttt{Flip()} trata de todos os casos necessários, é 
importante observar que os casos 5, 6 e 7 não são necessitam de tratamento, uma 
vez que nenhuma operação não é necessário. 

\subsection{Extraindo o DV}

Finalmente, seja $\mathcal{T}$ a $TD(S)$ em $\mathbb{R}^3$. Para extrair o DV
de $\mathcal{T}$ pode ser computado da seguinte forma:
\begin{itemize}
    \item[\textbf{Vértice: }] um vértice de Voronoi é exatamente o centro da esfera
    que passa pelos quatro vértices do tetraedro.
    \item[\textbf{Aresta: }] uma aresta de Voronoi, que é o dual da face triangular $k$,
    é formado pelos dois vértices de Voronoi cujo dual são as duas faces do tetraedro 
    que compartilham $k$.
    \item[\textbf{Face: }] a face de Voronoi, que é o dual a uma aresta de Delaunay $\alpha$,
    é formado por todos os vértices que são duais ao tetraedro de Delaunay incidente 
    a $\alpha$. 
    \item[\textbf{Poliedro: }] a contrução de uma região de Voronoi $\mathcal{V}_p$ dual 
    para um vértice $p$ consiste em identificar as arestas incidente ao vértice $p$, e 
    então extrair a dual da face de cada aresta.
\end{itemize}

Para extrair o diagrama de Voronoi em $\mathbb{R}^3$ da tetraedralização de Delaunay
para um conjunto $S$ de $n$ pontos no espaço, executa em $\Theta(n)$.

\subsection{Complexidade de tempo}

O algoritmo baseado em \texttt{InsertOnePoint()} é usado para computar a $TD(S)$
de um conjunto $S$ de $n$ pontos no $\mathbb{R}^3$, executa em $\mathcal{O}(n^2)$ \cite{Ledoux2007},
que é o pior caso ótimo \cite{Edelsbrunner1992}. Mas na prática, o algoritmo 
executará mais rápido \cite{Ledoux2007}. Inclusive pode executar em $\mathcal{O}(n\log n)$,
caso os $n$ pontos estiverem uniformemente distribuidos em um cubo unitário \cite{Edelsbrunner1992}.
Então, a complexidade de tempo total para se computar o diagrama de Voronoi em $\mathbb{R}^3$
é $\mathcal{O}(n^2)$.


