\section{Introdução}

O diagrama de Voronoi foi primeiro descrito, ainda que implicitamente, por 
R. Descartes \cite{descartesPP}, ao dizer que o sistema solar consiste de vórtices.
Ele afirmou que um corpo no espaço sofria maior influencia pela estrela que estivesse
mais próxima daquele corpo. Assim, cada estrela teria uma região convexa de influência.
Estas regiões convexas descritas por Descartes são as chamadas regiões que formam
o \textit{diagrama de Voronoi}. 

Este conceito se provou útil em várias áreas da ciência. Cada área deu um nome
diferente para o diagrama de Voronoi. Os matemáticos Dirichlet \cite{LejeuneDirichlet1850} 
e Voronoi \cite{Voronoi1909} foram os primeiros a definir formalmente o conceito.
Por isso que o diagrama é também conhecido como \textit{Dirichlet tessellation}.

Voronoi foi o primeiro a considerar o \textit{dual} dessa estrutura, onde duas 
regiões estão conectadas se possuirem uma fronteira em comum. Mais tarde, Delaunay
\cite{Delaunay1934} conclui o mesmo definindo que duas regiões estão conectadas
\textit{se e somente se} os pontos dessas regiões estão contidos em um círculo
cujo interior não possui nenhum outro ponto de $S$. Por isso, o diagrama de Voronoi
também é conhecido por ser equivalente à \textit{Triangulação de Delaunay}.
Ou seja, dado uma triangulação de Delaunay, é possível se obter o diagrama
de Voronoi facilmente.

No contexto da geometria computacional, os diagramas de Voronoi foram introduzidos
pela primeira vez por Hoey e Shamos \cite{ShamosHoey1975}, aprensentando um 
limite inferior para o problema e $O(n\log n)$. Existem alguns algoritmos na literatura
que resolvem o diagrama de Voronoi em $\mathbb{R}^2$ que possuem complexidade de $O(n\log n)$, 
um algoritmo que se destaca é o de Fortune \cite{Fortune1986} que 
utiliza a técnica de \textit{sweep line}. Porém, para resolver o diagrama de Voronoi em 
$\mathbb{R}^3$ é um pouco mais complicado. 

Este trabalho tem como objetivo dissertar sobre o diagrama de Voronoi em $\mathbb{R}^3$
especificamente, será apresentado algumas definições e propriedades básicas do problema.
Então, veremos como podemos calcular o diagrama de Voronoi a partir da Tetraedralização 
de Delaunay usando uma técnica de inserção incremental.
