\documentclass{article}

\newcommand{\senoide}{\mbox{$y = A\sin{(\omega x + \varphi)}$}}

\usepackage[utf8]{inputenc}

\title{Diagrama de Voronoi $\mathbb{R}^3$}
\author{Gustavo Higuchi}
\date{\today}

\usepackage[numbers]{natbib}
\usepackage{graphicx}
\usepackage{amssymb}
\usepackage{amsthm}
\usepackage{amsmath}
\usepackage{color}   %May be necessary if you want to color links
\usepackage[portuguese, ruled, linesnumbered]{algorithm2e}
\usepackage{float}
\usepackage{pgfplots}
\usepackage{enumitem}

\usepgfplotslibrary{fillbetween}
\pgfkeys{/pgfplots/Axis Style/.style={
    width=13.5cm, height=5cm,
    axis x line=center, 
    axis y line=middle, 
    samples=200,
    ymin=-1.5, ymax=1.5,
    xmin=0, xmax=13.0,
    domain=0:4*pi
}}

\newtheorem{teorema}{Teorema}
\theoremstyle{definition} 
\newtheorem{definicao}{Definição}


\newtheorem*{exemplo}{Exemplo}
\newlist{Casos}{enumerate}{1}
\setlist[Casos]{label=\textbf{\texttt{Caso }}\arabic*:}

\usepackage{mathtools}
\DeclarePairedDelimiter\ceil{\lceil}{\rceil}
\DeclarePairedDelimiter\floor{\lfloor}{\rfloor}

% usado para linkar cada section na tabela de conteúdo com a respectiva
% página no documento
\usepackage{hyperref}
\hypersetup{
    colorlinks,
    citecolor=black,
    filecolor=black,
    linkcolor=black,
    urlcolor=black,
    linktoc=all
}

%o começo do documento
\pgfplotsset{compat=1.13}
\begin{document}
% compila o título
\maketitle

% compila a tabela de conteúdos
\tableofcontents
\newpage

\section{Introdução}

O diagrama de Voronoi foi primeiro descrito, ainda que implicitamente, por 
R. Descartes \cite{descartesPP}, ao dizer que o sistema solar consiste de vórtices.
Ele afirmou que um corpo no espaço sofria maior influencia pela estrela que estivesse
mais próxima daquele corpo. Assim, cada estrela teria uma região convexa de influência.
Estas regiões convexas descritas por Descartes são as chamadas regiões que formam
o \textit{diagrama de Voronoi}. 

Este conceito se provou útil em várias áreas da ciência. Cada área deu um nome
diferente para o diagrama de Voronoi. Os matemáticos Dirichlet \cite{LejeuneDirichlet1850} 
e Voronoi \cite{Voronoi1909} foram os primeiros a definir formalmente o conceito.
Por isso que o diagrama é também conhecido como \textit{Dirichlet tessellation}.

Voronoi foi o primeiro a considerar o \textit{dual} dessa estrutura, onde duas 
regiões estão conectadas se possuirem uma fronteira em comum. Mais tarde, Delaunay
\cite{Delaunay1934} conclui o mesmo definindo que duas regiões estão conectadas
\textit{se e somente se} os pontos dessas regiões estão contidos em um círculo
cujo interior não possui nenhum outro ponto de $S$. Por isso, o diagrama de Voronoi
também é conhecido por ser equivalente à \textit{Triangulação de Delaunay}.
Ou seja, dado uma triangulação de Delaunay, é possível se obter o diagrama
de Voronoi facilmente.

No contexto da geometria computacional, os diagramas de Voronoi foram introduzidos
pela primeira vez por Hoey e Shamos \cite{ShamosHoey1975}, aprensentando um 
limite inferior para o problema e $O(n\log n)$. Existem alguns algoritmos na literatura
que resolvem o diagrama de Voronoi em $\mathbb{R}^2$ que possuem complexidade de $O(n\log n)$, 
um algoritmo que se destaca é o de Fortune \cite{Fortune1986} que 
utiliza a técnica de \textit{sweep line}. Porém, para resolver o diagrama de Voronoi em 
$\mathbb{R}^3$ é um pouco mais complicado. 

Este trabalho tem como objetivo dissertar sobre o diagrama de Voronoi em $\mathbb{R}^3$
especificamente, será apresentado algumas definições e propriedades básicas do problema.
Então, veremos como podemos calcular o diagrama de Voronoi a partir da Tetraedralização 
de Delaunay usando uma técnica de inserção incremental.


\section{Definições e propriedades básicas}

Para a definição do diagrama de Voronoi $DV(S)$ em $\mathbb{R}^3$, temos
\begin{definicao}
    Seja $S$ um conjunto de $n$ pontos no espaço Euclidiano $\mathbb{R}^2$. Uma região
    de Voronoi $V(p,S)$, é o conjunto de pontos $x \in \mathbb{R}^3$ que estão mais perto de 
    $p$ que qualquer outro ponto em $S$. A união das regiões $V(p,S)$ para todo $p \in S$ 
    formam o diagrama de Voronoi de $S$, denotado como $VD(S)$. Para três dimensões,
    $V(p,S)$ é um poliedro convexo. 
\end{definicao}

\begin{definicao}
    A triangulação de Delaunay $TD(S)$ em $\mathbb{R}^3$ consiste em separar o espaço em tetraedros tal que
    satisfaça a condição \textit{circumesfera vazia}, i.e., não possuir nenhum outro
    ponto dentro da esfera formada pelos vértices do tetraedro. 
\end{definicao}

O dual do diagrama de Voronoi em $\mathbb{R}^3$ é a \textit{tetraedralização de Delaunay}, 
cada elemento do diagrama de Voronoi corresponde a exatamente um elemento da 
tetraedralização de Delaunay. Assim, cada vértice de Delaunay corresponde à 
uma região de Voronoi, cada aresta de Delaunay corresponde a uma face de Voronoi,
cada face de Delaunay corresponde a uma aresta de Voronoi, e cada tetraedro de
Delaunay corresponde a um vértice de Voronoi.


\section{Operações básicas e estrutura de dados}

A computação da triangulação de Delaunay em duas e três dimensões pode
ser calculada basicamente de três formas: dividir e conquistar, \textit{sweep line}
e uma inserção incremental. Cada um desses paradigmas possui um algoritmo 
ótimo em duas dimensões, ou seja, pode ser calculado em $\mathcal{O}(n\log{}n)$.

Porém, em três dimensões não é tão simples, apenas algoritmos que utilizam 
o paradigma de inserção incremental possue um pior caso ótimo $\mathcal{O}(n^2)$,
uma vez que a complexidade da $TD$ em $\mathbb{R}^3$ é quadrático.

No caso de um algoritmo de inserção incremental, os pontos são inseridos um de 
cada vez e a tetraedralização é atualizada com relação ao critério de Delaunay.
Para isso, existem dois algoritmos na literatura que resolvem a TD de forma 
incremental: Bowyer-Watson \cite{Watson81}, ou um algoritmo baseado em 
\textit{flipping}. Nesta seção, será apresentado algumas operações básicas 
que serão necessárias para entender como que o algoritmo funciona e na 
proxima seção o algoritmo propriamente dito.

\subsection{Inicialização}
O algoritmo que será apresentadado nesta seção assume que o conjunto de 
pontos $S$ está inteiramente contido em um \textit{grande tetraedro} ($\tau_{big}$),
tal que o fecho convexo de $S$ seja $\tau_{big}$ exatamente. Portanto, inicia-se 
primeiro $\tau_{big}$ e em seguida insere-se os pontos de $S$ um a um, sem 
se preocupar se os pontos estão fora do fecho convexo. 

\subsection{Estrutura de dados}
A estrutura de dados comum de ser utilizada trata um tetraedro como se fosse 
um átomo, possuindo um ponteiro para os quatro vértices e um ponteiro 
para os quatro tetraedros adjacentes, por ser eficiente espaço
utilizado. Existem outras formas de estrutura de dados na literatura, inclusive
é possível guardar tanto a TD e o DV ao mesmo tempo \cite{Dobkin1987}.

\subsection{Predicados}
Para construir a TD é necessário dois testes básicos geométricos,
chamados \textit{predicados} \cite{Ledoux2007}.

O primeiro deles é $Orient(a, b, c, p)$ que determina se um ponto $p$ está acima,
abaixo ou diretamente no plano definido por três pontos $a$, $b$, e $c$;
\begin{equation}
    Orient(a,b,c,p) = 
        \begin{vmatrix}
            a_x & a_y & a_z & 1 \\
            b_x & b_y & b_z & 1 \\
            c_x & c_y & c_z & 1 \\
            p_x & p_y & p_z & 1
        \end{vmatrix}
\label{orient}
\end{equation}

O segundo é $InSphere(a, b, c, d, p)$ que determina se o ponto $p$ está dentro, fora ou 
diretamente na superfície da esfera definida por quatro pontos, $a$, $b$, $c$ e
$d$. 
\begin{equation}
    InSphere(a, b, c, d, p) = 
        \begin{vmatrix}
            a_x & a_y & a_z & a_x^2 + a_y^2 + a_z^2 & 1  \\
            b_x & b_y & b_z & b_x^2 + b_y^2 + b_z^2 & 1 \\
            c_x & c_y & c_z & c_x^2 + c_y^2 + c_z^2 & 1 \\
            d_x & d_y & d_z & d_x^2 + d_y^2 + d_z^2 & 1 \\
            p_x & p_y & p_z & p_x^2 + p_y^2 + p_z^2 & 1
        \end{vmatrix}
\label{insphere}
\end{equation}

Ambos os testes podem ser reduzidos para a computação do determinante de uma matriz 
\cite{Guibas1985} como mostram as equações \ref{orient} e \ref{insphere}. Portanto,
ambos os predicados são implementados como determinantes de matrizes 3x3 e 4x4, onde
$Orient()$ retorna um número positivo caso o $p$ esteja acima do plano definido por
$a$, $b$ e $c$; 0 caso esteja diretamente no plano, e negativo caso esteja abaixo.
Semelhante para $InSphere()$, positivo caso o ponto $p$ esteja dentro da esfera 
definida por $a$, $b$, $c$ e $d$; 0 caso esteja diretamente na superficie da esfera e
negativo caso esteja fora da esfera.

\subsection{``Bistellar Flips" tridimensionais}

Um \textit{Bistellar Flip} é uma operação local que modifica a configuração de alguns 
tetraedros adjacentes \cite{Lawson1987, Edelsbrunner1992}.

Para entender, considere o conjunto $S = \{a,b,c,d,e\}$ de pontos em $\mathbb{R}^3$ e seu
fecho convexo $conv(S)$. Disso, temos duas possibilidades:
\begin{enumerate}
    \item todos os pontos de $S$ estão na fronteira de $conv(S)$ e, segundo Lawson \cite{Lawson1987},
    existem duas formas de \textit{``tetraedralizar"} o poliedro, em dois ou três tetraedros. 
    \item ou o ponto $e$ de $S$ está contido em $conv(S)$, então existe apenas uma forma de 
    ``tetraedralizar" $S$, que seria formar quatro tetraedros incidentes em $e$.
\end{enumerate}

Baseado nessas duas configurações descritas acima, temos quatro tipos de $flips$, são eles:
$flip23$, $flip32$, $flip14$ e $flip41$ (os números indicam a quantidade de tetraedros 
antes e depois, respectivamente). Para o primeiro caso, temos dois $flips$ possíveis de executar
$flip23$ que consiste em transformar dois tetraedros em três, e $flip32$ que é a operação inversa.
Caso $S$ seja ``tetraedralizado" em dois tetraedros e a face triangular $bcd$ não for Delaunay
localmente, então um $flip23$ gera três tetraedros cujas faces são Delaunay localmente.

Um $flip14$ acontece quando um ponto de $S$ é inserido dentro de um tetraedro, então
divide um tetraedro em quatro, e o $flip41$ é o processo inverso quando se remove o
ponto.

\subsubsection{Flips para casos especiais}

Existem alguns $flips$ baseados em casos especiais que são definidos e usados
por Shewmuck \cite{Shewchuk2003}. São eles:
\begin{enumerate}
    \item $flip12$ que separa um tetraedro $abcd$ em dois tetraedros quando um novo
    ponto é inserido diretamente na aresta $\overline{ac}$, por exemplo.
    \item $flip13$ que separa um tetraedro $abcd$ em três tetraedros quando um novo
    ponto é inserido diretamente em uma face do tetraedro.
    \item $flip44$ que é equivalente a um $flip23$, formando um tetraedro achatado,
    seguido imediatamente de um $flip32$, que remove o tetraedro achatado. 
\end{enumerate}

\subsection{Localização de ponto}

A última operação básica para se calcular a $TD$ é dizer em qual tetraedro um ponto
$p$ recem inserido está dentro. Ou seja, dado uma $TD$ $\mathcal{T}$ de um conjunto
$S$ de pontos e um ponto $p$, dizer em qual tetraedro de $\mathcal{T}$ o ponto $p$
pertence.

Para isso, o algoritmo que utiliza a estratégia ``walking" (denominado $Walk$ no algoritmo) 
que, baseados em resultados experimentais de Mücke et al. \cite{Mucke1996} e de Devillers 
et al. \cite{Devillers2001}, é a solução mais rápida para caminhar na TD, também é
trivial manter $Walk$ robusto, por não ser afetado em casos especiais.



\section{Algoritmo Incremental baseado em flips}

O algoritmo que será apresentado é baseado na inserção dos pontos de $S$ um de cada vez.
Para cada novo ponto $p$ inserido, é modificado a configuração dos tetraedros presentes nas vizinhanças 
de $p$ através de $flips$ que foram descritos acima. 

O algoritmo baseado em flips é relativamente mais demorado que o algoritmo que 
gera um buraco, apresentado por \cite{Engwirda2015}, porém é mais simples de implementar.

\subsection{Duas dimensões}

O primeiro algoritmo baseado em flips foi feito para construir a TD em duas dimensões.
Desenvolvido por Lawson \cite{Lawson1977} que provou que, começando com uma 
triangulação arbitrária de um conjunto $S$ de pontos no plano, ``flipping" 
as arestas pode transformar essa triangulação em qualquer outra triangulação de S,
incluindo uma triangulação de Delaunay. 

Basicamente, primeiro o algoritmo busca o triangulo $\tau$ que contém o ponto
$p$ recem inserido e então o divide em três novos triangulos que contém o 
ponto $p$. Em seguida, os novos triangulos gerados são testados com os triangulos
opostos em relação à $p$. Caso algum triangulo não seja um triangulo de Delaunay,
então a aresta compartilhada pelos triangulos é ``flipado" e os dois novos triangulos
gerados desta forma serão testados depois. 

\subsection{Três dimensões}

Apesar do conceito de ``flipping" generalizar para três dimensões, o algoritmo 
de Lawson \cite{Lawson1977} não generaliza, provado por Joe \cite{Joe1989}.
Mais tarde, porém, Joe \cite{Joe1991} prova que dado uma $TD(S)$ e um ponto 
$p$ em $\mathbb{R}^3$, sempre existe pelo menos uma sequencia de $flips$ para 
construir a $TD(S \cup \{p\})$. Neste caso, ainda haverá facetas que não são
Delaunay que não se pode aplicar um \textit{flips}, mas que sempre haverá algum
$flip$ possível de se executar tal que o algoritmo continue executando.

Então, com isso é possível adaptar o algoritmo de duas dimensões para que 
se gere o $TD(S)$ em três dimensões. Daí temos 
\begin{definicao}
    Seja $S$ um conjunto de pontos em $\mathbb{R}^3$, e $\mathcal{T}$ a $TD(S)$,
    o algoritmo \texttt{InsertOnePoint()} é necessário para manter $\mathcal{T}$
    dentro dos critérios de Delaunay quando um ponto $p$ é inserido.    
\end{definicao}

\begin{algorithm}
\DontPrintSemicolon % Some LaTeX compilers require you to use \dontprintsemicolon instead
\KwIn{Uma $TD(S) \mathcal{T}$ em $\mathbb{R}^3$, e um novo ponto $p$ para inserir}
\KwOut{$\mathcal{T}^p = \mathcal{T} \cup \{p\}$}
$\tau \gets Walk(p)$\\
$\text{insere p em }\tau\text{ com um flip14}$\\
$\text{empilha os 4 tetraedros que se formaram}$\\
\While{$\text{pilha não for vazia}$}{
  $\tau = \{p,a,b,c\} \gets \text{desempilha um tetraedro}$\\
  $\tau_a = \{a,b,c,d\} \gets \text{tetraedro adjacente de }\tau\text{ que contém abc como faceta}$\\
  \uIf{$\text{d está dentro da circunsfera de }\tau$}{
      $Flip(\tau, \tau_a)$\\
    }
  }
\caption{\texttt{InsertOnePoint($\mathcal{T}, p$)}}
\label{alg1}
\end{algorithm}

O algoritmo é uma adaptação de \cite{Joe1991}. Como em $\mathbb{R}^2$, o ponto $p$
é primeiro inserido em $\mathcal{T}$ com um $flip$ (neste caso, $flip14$), e os 
novos tetraedros são testados para garantir que ainda são Delaunay. A sequencia de 
$flips$ necessários é controlado pela pilha que contém todos os tetraedros não
testados. Toda vez que $flip$ é executado, o tetraedro gerado é empilhado. O algoritmo
para quando todos os tetraedros incidentes a $p$ são Delaunay, o que significa que 
a pilha está vazia.

O método \texttt{Flip()} necessita de cuidado especial, ao contrário do caso para duas
dimensões, existem diferentes tipos de $flips$ para diferentes situações. Basicamente,
testar para cada face que não possue o ponto inserido $p$, se são localmente 
Delaunay.  

Seja $\tau = \{p,a,b,c\}$ o próximo tetraedro a ser desempilhado. Para que 
$\tau$ seja Delaunay, é necessário apenas uma comparação com o tetraedro 
$\tau_a = \{a,b,c,d\}$ que não possue $p$ e é incidente à face $abc$. Se o
vértice $d$ estiver dentro da circunsfera de $\tau$. Caso esteja, realiza
uma operação diferente dependendo do tipo de geometria que de $\tau$ e 
$\tau_a$.

Observando do ponto de vista do ponto $p$, temos três possibilidades: vemos
uma face de $\tau_a$; vemos duas faces de $\tau_a$; ou vemos três faces de
$\tau_a$. Joe \cite{Joe1989} não só provou que o algoritmo incremental baseado
em $flip$ que funciona em 3 dimensões, mas também provou que funciona para
casos degenerados. Disso temos os seguintes casos que podem ocorrer:

\begin{Casos}
    \item apenas uma face de $\tau_a$ é visível, e portanto a união de 
    $\tau$ e $\tau_a$ é um poliedro convexo. Neste caso, executa um 
    $flip23$.
    \item duas faces de $\tau_a$ são visíveis, e portanto a união de 
    $\tau$ e $\tau_a$ é não-convexo. \textbf{Se} existir um tetraedro $\tau_b = abpd$
    em $\mathcal{T}^p$ tal que a aresta $ab$ é compartilhada por $\tau$,
    $tau_a$ e $\tau_b$, então um $flip32$ é executado. \textbf{Caso contrário},
    nenhum $flip$ é executado. A face $abc$ que não é Delaunay será corrigida
    por outro $flip$ executado em um tetraedro adjacente, como \cite{Joe1991}
    prova.
    \item o segmento de reta $pd$ intersecta diretamente uma aresta de $abc$ 
    (assuma que seja $ab$, por exemplo). Então, os vértices $p$, $d$, $a$ e 
    $b$ são coplanares. Um $flip23$ criaria um tetraedro ``achatado" $abdp$,
    é possível se, e somente se, $\tau$ e $\tau_a$ estão na $config44$ com dois
    outros tetraedros $\tau_b$ e $\tau_c$, tal que a aresta $ab$ é incidente a
    $\tau$, $\tau_a$, $\tau_b$ e $\tau_c$. Uma vez que os quatro tetraedros estão
    na $config44$, um $flip44$ pode ser executado, caso contrário, nenhum $flip$
    é executado.
    \item o vértice $p$ está diretamente em cima de uma aresta da face $abc$ (assuma 
    que seja a aresta $ab$). Portanto, $p$ está no mesmo plano que abc, mas também
    está no mesmo plano que $abd$. O tetraedro $abcp$ é obviamente achatado. A única 
    razão para tal caso é porque $p$ foi inserido diretamente em uma aresta de $\mathcal{T}$,
    e o $flip14$ que divide o tetraedro, criou dois tetraedros achatados que contém $p$.
    Neste caso, é suficiente executar um $flip23$ em $\tau$ e $\tau_a$. Um tetraedro 
    achatado ainda será criado, porém será deletado com um outro $flip$.
    \item existe um ponto exatamente no mesmo lugar que o ponto $p$ que está sendo inserido,
    este caso é trivial, no método \texttt{Walk()}, testar se o ponto $p$ não está
    dentro de uma tolerancia de distancia dos vértices, caso esteja, não insere $p$.
    \item o ponto $p$ cai exatamente na superfície de uma circunsfera de $\mathcal{T}$,
    este caso também é trivial e não necessita de nenhuma operação.
    \item o ponto $p$ cai em uma face de um tetraedro de $\mathcal{T}$. Neste caso, 
    não necessita de nenhuma operação, uma vez que um $flip14$ será executado e criará
    um tetraedro achatado que será deletado mais tarde. 
\end{Casos}

\begin{algorithm}[H]
\DontPrintSemicolon % Some LaTeX compilers require you to use \dontprintsemicolon instead
\KwIn{Dois tetraedros adjacentes $\tau$ e $\tau_a$}
\KwOut{um flip é executa, ou não}
\uIf{\text{caso 1}}{
    $flip23(\tau, \tau_a)$\\
    $\text{empilha tetraedra $pabd$, $pbcd$ e $pacd$}$
}
\Else{
    \uIf{\text{caso 2 }\textbf{and}\text{ $\mathcal{T}^p$ tem tetraedro $pdab$} }{
        $flip32(\tau, \tau_a, pdab)$\\
        \text{empilha $pacd$ e $pbcd$}
    }
    \Else{
        \uIf{\text{caso 3 }\textbf{and}\text{ $\tau$ e $\tau_a$ estão em $config44$ com $\tau_b$ e $\tau_c$}}{
            $flip44(\tau, \tau_a, \tau_b, \tau_c)$\\
            \text{empilha os quatro tetraedros criados}
        }
        \Else{
            \uIf{\text{caso 4}}{
                $flip23(\tau, \tau_a)$\\
                \text{empilha o tetraedro $pabd$, $pbcd$ e $pacd$}
            }
        }
    }
}
\caption{\texttt{Flip($\mathcal{T}, p$)}}
\label{alg1}
\end{algorithm}

O algoritmo \texttt{Flip()} trata de todos os casos necessários, é 
importante observar que os casos 5, 6 e 7 não são necessitam de tratamento, uma 
vez que nenhuma operação não é necessário. 

\subsection{Extraindo o DV}

Finalmente, seja $\mathcal{T}$ a $TD(S)$ em $\mathbb{R}^3$. Para extrair o DV
de $\mathcal{T}$ pode ser computado da seguinte forma:
\begin{itemize}
    \item[\textbf{Vértice: }] um vértice de Voronoi é exatamente o centro da esfera
    que passa pelos quatro vértices do tetraedro.
    \item[\textbf{Aresta: }] uma aresta de Voronoi, que é o dual da face triangular $k$,
    é formado pelos dois vértices de Voronoi cujo dual são as duas faces do tetraedro 
    que compartilham $k$.
    \item[\textbf{Face: }] a face de Voronoi, que é o dual a uma aresta de Delaunay $\alpha$,
    é formado por todos os vértices que são duais ao tetraedro de Delaunay incidente 
    a $\alpha$. 
    \item[\textbf{Poliedro: }] a contrução de uma região de Voronoi $\mathcal{V}_p$ dual 
    para um vértice $p$ consiste em identificar as arestas incidente ao vértice $p$, e 
    então extrair a dual da face de cada aresta.
\end{itemize}

Para extrair o diagrama de Voronoi em $\mathbb{R}^3$ da tetraedralização de Delaunay
para um conjunto $S$ de $n$ pontos no espaço, executa em $\Theta(n)$.

\subsection{Complexidade de tempo}

O algoritmo baseado em \texttt{InsertOnePoint()} é usado para computar a $TD(S)$
de um conjunto $S$ de $n$ pontos no $\mathbb{R}^3$, executa em $\mathcal{O}(n^2)$ \cite{Ledoux2007},
que é o pior caso ótimo \cite{Edelsbrunner1992}. Mas na prática, o algoritmo 
executará mais rápido \cite{Ledoux2007}. Inclusive pode executar em $\mathcal{O}(n\log n)$,
caso os $n$ pontos estiverem uniformemente distribuidos em um cubo unitário \cite{Edelsbrunner1992}.
Então, a complexidade de tempo total para se computar o diagrama de Voronoi em $\mathbb{R}^3$
é $\mathcal{O}(n^2)$.




\section{Conclusão}

Ainda há outros pontos que não foram mencionados como por exemplo
a precisão do cálculo de \texttt{InSphere()} e \texttt{Orient()}, 
que possivelmente pode ocorrer resultados errados. Existem formas
de realizar operações exatas em ponto flutuante, porém não são
algoritmos rápidos \cite{Yap94theexact}. Ledoux explica que a 
abordagem apresentada neste documento necessita dessa exatidão,
mas não explica o porquê \cite{Ledoux2007}. Outros pontos não 
mencionados são por si só tópicos de pesquisas. Por fim, vale 
mencionar que existem bibliotecas que possuem implementações 
do diagrama de Voronoi em três dimensões disponíveis gratuitamente
na internet, como \texttt{CGAL}. 

\newpage
\bibliography{voronoi3d}
\bibliographystyle{ieeetr}
\end{document}

