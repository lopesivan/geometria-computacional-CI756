\documentclass{article}
\usepackage[utf8]{inputenc}

\title{Trabalho 2 - DCEL 3D}
\author{Gustavo Higuchi}
\date{\today}

\usepackage{natbib}
\usepackage{graphicx}
\usepackage{amssymb}
\usepackage{amsthm}
\usepackage{amsmath}
\usepackage{color}   %May be necessary if you want to color links
\usepackage[portuguese, ruled, linesnumbered]{algorithm2e}
\usepackage{inconsolata}

\usepackage{mathtools}
\DeclarePairedDelimiter\ceil{\lceil}{\rceil}
\DeclarePairedDelimiter\floor{\lfloor}{\rfloor}

% usado para linkar cada section na tabela de conteúdo com a respectiva
% página no documento
\usepackage{hyperref}
\hypersetup{
    colorlinks,
    citecolor=black,
    filecolor=black,
    linkcolor=black,
    urlcolor=black,
    linktoc=all
}

%o começo do documento
\begin{document}

% compila o título
\maketitle
\newpage
% compila a tabela de conteúdos
\tableofcontents
\newpage
\section{Resumo}
\hspace*{15pt}Sem muita complexidade, o algoritmo $dcel$ descrito neste documento
gera a representação de um poliedro com a estrutura de dados DCEL.

\section{Introdução}
\hspace*{15pt}A estrutura de dados DCEL apresentada no livro
\textit{Computational Geometry Algorithms and Applications} \cite{bergBook} é
adaptada para representar um poliedro (figura 3D).

\section{Descrição do problema}
\hspace*{15pt}Implementar uma estrutura de dados DCEL (\textit{Doubly Connneted Edge List})
para poliedros.

\section{DCEL 3D}

\hspace*{15pt}A estrutura de dados gerada é a mesma que do livro da disciplina,
\textit{Doubly Connneted Edge List} ou \textit{DCEL} e é descrita no livro de
Berg et al. \cite{bergBook}, porém para poliedros:
\begin{enumerate}
    \item Cada vértice possui sua coordenada e aponta para uma aresta.
    \item Cada face possui um apontador para uma aresta interna
    \item Cada aresta possui um apontador para o vértice de origem, a face esquerda,
          a próxima aresta e a aresta anterior.
    \item O poliedro possui uma lista de vértices, uma lista de arestas e uma lista de
          faces.
\end{enumerate}

\hspace*{15pt} No algoritmo abaixo, a função $aresta\_compartilhada(v_{i-1}, v_i)$ procura
uma semi-aresta que possui do vértice $v_{i-1}$ para o $v_i$.\\
\\

\begin{algorithm}[H]
    \SetAlgoLined
    \Entrada{um conjunto de vértice $V$, um conjunto de faces $F$, e os vértices de cada face $FV$ }
    \Saida{O poliedro $P$}
    \Inicio{
    $Q\gets \text{lista de faces de P}$\\
    \Para{\text{cada face $f$ em $F$}}{
        \Para{cada vertice em $FV$}{
            $e \gets aresta\_compartilhada(v_{i-1}, v_i)$\\
            \Se{ \text{$e$ existir} }{
                \Se{\text{possuir uma face esquerda definida}}{
                    \Retorna $\text{não é orientável}$\\
                }
                $e.esquerda \gets f$\\
            }
            \Senao{
                $\text{crio a semi-aresta e sua simétrica}$\\
                $\text{adiciono ao poliedro}$\\
            }
        }
        \Para{aresta da face}{
            $\text{defino quem é proximo e anterior}$\\
        }
        \Se{$f$ ainda não possui uma aresta interna}{
            $f.aresta \gets \text{última semi-aresta criada}$\\
        }
    }
    }
    \label{alg1}
    \caption{dcel}
\end{algorithm}

\subsection{Corretude}
\hspace*{15pt}Para gerar a estrutura, a função \textit{dcel} descrita acima,
recebe os vértices, as faces e seus respectivos vértices. Assim, para cada
face, sempre cria duas semi-arestas, a semi-aresta e sua simétrica, para cada
par de vértices daquela face. Caso já exista uma semi-aresta com o par de vértice,
não cria nenhuma semi-aresta. Após a criação de todas as semi-arestas da face,
define a semi-aresta próxima e anterior para cada uma delas.
\\
\newpage

\bibliography{relatorio}
\bibliographystyle{ieeetr}

\end{document}
